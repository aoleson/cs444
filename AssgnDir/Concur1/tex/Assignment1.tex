\documentclass[letterpaper,10pt,titlepage]{article}

\usepackage{hyperref}
\usepackage{titling}

\usepackage{graphicx}                                        
\usepackage{amssymb}                                         
\usepackage{amsmath}                                         
\usepackage{amsthm}                                          

\usepackage{alltt}                                           
\usepackage{float}
\usepackage{color}
\usepackage{url}

\usepackage{balance}
\usepackage{pstricks, pst-node}

\usepackage{geometry}
\usepackage{listings}  
\lstset{language=C}
\geometry{textheight=8.5in, textwidth=6in}

%random comment

\newcommand{\cred}[1]{{\color{red}#1}}
\newcommand{\cblue}[1]{{\color{blue}#1}}

\usepackage{hyperref}
\usepackage{geometry}

\def\name{Alannah Oleson, Jacob Mahugh}


%% The following metadata will show up in the PDF properties
\hypersetup{
  colorlinks = true,
  urlcolor = black,
  pdfauthor = {\name},
  pdfkeywords = {cs444},
  pdftitle = {CS 444 Project 1: Getting Acquainted},
  pdfsubject = {CS 444 Project 1},
  pdfpagemode = UseNone
}


\title{Project 1 - Getting Acquainted}
\title{CS 444 - Operating Systems 2}
\title{Fall 2017}
\author{\name}


\begin{document}


\begin{titlepage}
   \title{Project 1 - Getting Acquainted\\CS 444\\Fall 2017}
   \author{\name}
   \maketitle

   \begin{abstract}
      In this assignment, we got our first look at how to build a Linux kernel and run it in a virtual machine.
      We set up our environment on the OS2 server and ensured that we could boot correctly.
      We then examined the flags of the given qemu command.
      Finally, we completed and analyzed our first concurrency assignment, learning how to leverage different threads to communicate through consumption and production of values.
   \end{abstract}

\end{titlepage}


\section{Command Line Commands}

\subsection{In Terminal 1}

\begin{itemize}
   \item \verb|cd /scratch/fall2017/30|
   \item \verb|git clone git://git.yoctoproject.org/linux-yocto-3.19|
   \item \verb|git branch v3.19.2|
   \item \verb|cp /scratch/files/environment-setup-i586-poky-linux .|
   \item \verb|cp /scratch/files/bzImage-qemux86.bin ./linux-yocto-3.19|
   \item \verb|cp /scratch/files/core-image-lsb-sdk-qemux86.ext4 ./linux-yocto-3.19|
   \item \verb|source environment-setup-i586-poky-linux|
   \item \verb|make -j4 all|
   \item \verb|qemu-system-i386 -gdb tcp::5530 -S -nographic -kernel bzImage-qemux86.bin -drive file=core-image-lsb-sdk-qemux86.ext4,if=virtio -enable-kvm -net none -usb -localtime --no-reboot --append "root=/dev/vda rw console=ttyS0 debug"|
\end{itemize}

\subsection{In Terminal 2}
\begin{itemize}
   \item \verb|gdb|
   \item \verb|file vmlinux|
   \item \verb|target remote :5530|
   \item Log in with root and no password.
\end{itemize}



\section{QEMU Command Line Flags}

\begin{itemize}
   \item gdb tcp::5530 : Open a gdb server on port 5530
   \item S : Don’t start CPU immediately on startup
   \item nographic : This allows us to debug the kernel without a VGA output (i.e. using the console instead). Usually QEMU uses SDL to output VGA, but using this flag disables graphical output entirely so that we can interact with it using only the command line
   \item kernel bzImage-qemux86.bin : Define the image to be used for the kernel (in this case,  bzlmage-qemux86.bin)
   \item drive file=core-image-lsb-sdk-qemux86.ext4, : Boot from something other than a CD-ROM (in this case, the file given – core-image-lib-sdk-qemux86.ext4)
   \item if=virtio : Check if the virtio driver is available. If it is, we can use KVM
   \item enable-kvm :  Enable KVM virtualization support. This only happens if KVM is actually supported while compiling (which we know from the previous command)
   \item net none : Overrides the default configuration of -net nic -net user. This changes qemu such that no network devices should be configured
   \item usb : Adds the USB device
   \item localtime : Use the local time
   \item no-reboot : Specifies that the system should exit instead of rebooting
   \item append "root=/dev/vda rw console=ttyS0 debug" : Command line arguments for the kernel. In this case, it’s specifying the path for the root and various console configurations
\end{itemize}


\section{Concurrency Questions}

\subsection{What do you think the main point of this assignment is?}

\subsection{How did you personally approach the problem? Design decisions, algorithm, etc.}

\subsection{How did you ensure your solution was correct? Testing details, for instance.}

\subsection{What did you learn?}


\section{Version Control Log} 

\begin{table}[ht]
\centering
\resizebox{\textwidth}{!}{
\begin{tabular}{l l l}\textbf{Detail} & \textbf{Author} & \textbf{Description}\\\hline
\href{git@github.com:aoleson/cs444/commit/eacda375043af0c890654149a289a781c0d558cf}{eacda37} & aoleson & initial commit. These files should compile and run the kernel and VM\\\hline
\href{git@github.com:aoleson/cs444/commit/4abbb240a90bdba422c608f9917f45b501decaac}{4abbb24} & Jacob Mahugh & Implemented Random Number Generation (RDRand or Mersenne Twister)\\\hline
\href{git@github.com:aoleson/cs444/commit/f40806a12aac697608b0364de9c29d803cbb688f}{f40806a} & Jacob Mahugh & Added queue implementation w/ custom struct\\\hline
\href{git@github.com:aoleson/cs444/commit/9887b1bce65d52ad32789073e14cdbdf505287cb}{9887b1b} & aoleson & Updating the writeup files. Added a template and a working Makefile\\\hline
\href{git@github.com:aoleson/cs444/commit/4e6c64d648c1ebccc697c451c50de5077be7ca20}{4e6c64d} & aoleson & Added a script to generate version control logs for the TeX doc. Modified the Tex files and Makefile accordingly.\\\hline
\href{git@github.com:aoleson/cs444/commit/8b7ecc3f64e5c56b29ca5f6f38a7992d4476c72c}{8b7ecc3} & aoleson & Added title page to writeup. Abstract is still on a still on a separate page, will need to modify to put it on the first page.\\\hline
\href{git@github.com:aoleson/cs444/commit/efe3633ebe7f21d53cfa58ea1648db660104e052}{efe3633} & Jacob Mahugh & Finalized file versions; trimmed out unnecessary functions, created makefile\\\hline
\href{git@github.com:aoleson/cs444/commit/a881cee4d8422c40bff2419b07f2f16a1456b713}{a881cee} & aoleson & Filled out first section of the writeup and added skeleton for the rest\\\hline
\href{git@github.com:aoleson/cs444/commit/e5676a9bdf5e22732146b240935abaa426459cfb}{e5676a9} & aoleson & Added second section of writeup\\\hline
\href{git@github.com:aoleson/cs444/commit/2302d58153455c6b6d444f4b20aceac4f336a5e5}{2302d58} & aoleson & Added third section of writeup\\\hline
\href{git@github.com:aoleson/cs444/commit/f7fc54a8ac2e09a8b5b0b627e774e24a883b106c}{f7fc54a} & aoleson & Added fourth section of writeup\\\hline
\href{git@github.com:aoleson/cs444/commit/62b683307c862aee0d7a79edbf93379de03cd2e2}{62b6833} & aoleson & Modified git log formatting\\\hline
\href{git@github.com:aoleson/cs444/commit/55a64e9e1580247f4e44542a55f0605072328469}{55a64e9} & aoleson & Changed a bunch of formatting\\\hline\end{tabular}}
\end{table}



\section{Work Log}


\end{document}

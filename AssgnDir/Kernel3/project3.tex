\documentclass[letterpaper,10pt,titlepage]{article}

\usepackage{graphicx}                                        
\usepackage{amssymb}                                         
\usepackage{amsmath}                                         
\usepackage{amsthm}                                          

\usepackage{alltt}                                           
\usepackage{float}
\usepackage{color}
\usepackage{url}

\usepackage{balance}
\usepackage{enumitem}
%\usepackage{pstricks, pst-node}

\usepackage{geometry}

% Code listing and colors
\usepackage{listings}  
\usepackage{color}
\lstset{
	language=C,
	basicstyle=\ttfamily,
	keywordstyle=\color{blue}\ttfamily,
	stringstyle=\color{red}\ttfamily,
	commentstyle=\color{green}\ttfamily,
	morecomment=[l][\color{magenta}]{\#}
}
% End code listing setup

\geometry{textheight=8.5in, textwidth=6in}

%random comment

\newcommand{\cred}[1]{{\color{red}#1}}
\newcommand{\cblue}[1]{{\color{blue}#1}}

\usepackage{hyperref}
\usepackage{geometry}

\def\name{Alannah Oleson, Jacob Mahugh}


%% The following metadata will show up in the PDF properties
\hypersetup{
  colorlinks = true,
  urlcolor = black,
  pdfauthor = {\name},
  pdfkeywords = {},
  pdftitle = {CS 444 Assignment 3},
  pdfsubject = {CS 444 Assignment 3},
  pdfpagemode = UseNone
}

\begin{document}

\begin{titlepage}
	\centering
	\vspace*{4cm}
	{\scshape\huge Project 3: Encrypted Block Device\par}
	\vspace{1cm}
	{\scshape\LARGE CS 444: Operating Systems II\par}
	\vspace{0.5cm}
	{\large\bfseries Fall 2017\par}
	{\large Abstract\par}
	\vspace {0.5cm}
	In the third kernel assignment, we implement an encrypted RAM block device
	as a kernel module and load it into our qemu VM at runtime.
	This document covers our implementation's changes,
	including how to boot up the kernel and ensure that the encryption is running
	correctly. This document also discusses our philosophy
	in designing the solution and what we learned from the process.
	\par
	\vspace{1cm}
	{\Large\itshape Jacob Mahugh\par}
    \vspace {0.5cm}
    {\Large\itshape Alannah Oleson\par}
	\vfill
	{\large \today\par}	
\end{titlepage}



\section{Program Design}
We began this assignment by looking at the LDD3 implementation linked on the class webpage.
Specifically, we found the most useful chapter to be Ch. 16, which described block device drivers. 
This chapter walked through each of the functions in an implementation of the SBULL block driver and how to write each of them. 
This gave us a solid foundation of knowledge on which to build our own driver.

Because of this, we decided to base our implementation off the working SBULL driver.
We found a base driver (at http://blog.superpat.com/2010/05/04/a-simple-block-driver-for-linux-kernel-2-6-31/) that we began with.

**TODO: finish**



\section{Version Control and Work Log}

**TODO: this is from last assignment, update**


\begin{table}[H]
\centering
\begin{tabular}{|p{3cm}|p{2cm}|p{2cm}|p{3cm}|p{2cm}|}
\hline
Description                                                                                              & Start time       & End (commit) time              & Commit number (if relevant)              & Author       \\
\hline
Cleaning up the kprint statements                                                                        & Mon Oct 30 17:00 & Mon Oct 30 18:02:06 2017 -0700 & 63aff2f837551 dcb576b7d0dd262 e0a953f17ed9 & aoleson      \\
\hline
cleaning up yocto directory target files                                                                 & Mon Oct 30 16:35 & Mon Oct 30 16:42:32 2017 -0700 & 67d36e7c4f922 20340a90f0831aa d458b3473194 & aoleson      \\
\hline
Committing full yocto directory. Ensures we get all the changes from the menuconfig options              & Mon Oct 30 16:30 & Mon Oct 30 16:31:46 2017 -0700 & 7eb6b4a42c3a8 6088ffe19119229 53076afaba98 & aoleson      \\
\hline
Tested working sstf files. Need to clean up kprints though                                               & Mon Oct 30 11:00 & Mon Oct 30 16:28:35 2017 -0700 & b4552cff631ae ebe7cff247b8892 8565d5dcfd01 & aoleson      \\
\hline
Added intial sstf-iosched.c as cp of noop                                                                & Fri Oct 27 10:00 & Fri Oct 27 10:52:56 2017 -0700 & 6e64485efc30e f2607f20b1c5ea5 8055b47971b8 & Jacob Mahugh \\
\hline
Basic file structure for concurreny asgn 2; notes on Arbitrator(waiter) solution implementation in code. & Thu Oct 26 07:00 & Thu Oct 26 09:10:39 2017 -0700 & ca40083ddeb61 780e828f0c19f92 d11cc997268c & Jacob Mahugh \\
\hline
Met up to design concurrency assignment 2 and kernel project 2; decided on CLOOK vs LOOK                 & Wed Oct 25 19:00 & Wed Oct 25 21:00               &                                          &             	 \\
\hline
\end{tabular}
\end{table}

\section{Questions}

\subsection{What do you think the main point of this assignment is?}


\subsection{How did you personally approach the problem?}


\subsection{How did you ensure your solution was correct?}


\subsection{What did you learn?}


\subsection{How should the TA evaluate your work?}
To evaluate our work, perform the following steps:

\begin{enumerate}

\end{enumerate}




%\section{Code Listing Test}

%\begin{lstlisting}
%#include <stdio.h>
%#define N 10
%/* Block
%* comment */

%int main()
%{
%	int i;
%
%	// Line comment.
%	puts("Hello world!");
%
%	for (i = 0; i < N; i++)
%	{
%		puts("LaTeX is also great for programmers!");
%	}
%
%	return 0;
%}
%\end{lstlisting}


\end{document}
